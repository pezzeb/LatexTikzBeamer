%%ENVIROMENTS
\newtheorem{mytheorem}{Theorem}
	\newtheorem*{mytheorem*}{Theorem}
		\newcommand{\mytheoremautorefname}{proposition}
\newtheorem{myproof}{Proof}
	\newtheorem*{myproof*}{Proof}
		\newcommand{\myproofautorefname}{proposition}
\newtheorem{mylemma}{Lemma}
	\newtheorem*{mylemma*}{Lemma}
		\newcommand{\mylemmaautorefname}{proposition}
\newtheorem{mycor}{Corollary}
	\newtheorem*{mycor*}{Corollary}
		\newcommand{\mycorautorefname}{proposition}
\newtheorem{mydef}{Definition}
	\newtheorem*{mydef*}{Definition}
		\newcommand{\mydefautorefname}{proposition}
\newtheorem{myexample}{Example}
	\newtheorem*{myexample*}{Example}
		\newcommand{\myexampleautorefname}{proposition}
\newtheorem{mycomment}{Comment}
	\newtheorem*{mycomment*}{Comment}
		\newcommand{\mycommentautorefname}{proposition}
\newtheorem{mymoral}{Moral}
	\newtheorem*{mymoral*}{Moral}
		\newcommand{\mymoralautorefname}{proposition}
\newtheorem{mynote}{Note}
	\newtheorem*{mynote*}{Note}
		\newcommand{\mynoteautorefname}{proposition}
\newtheorem{myprop}{Proposition}
	\newtheorem*{myprop*}{Proposition}
		\newcommand{\mypropautorefname}{proposition}
\newtheorem{myass}{Assumption}
	\newtheorem*{myass*}{Assumption}
		\newcommand{\myassautorefname}{assumption}
\newtheorem{myremark}{Remark}
	\newtheorem*{myremark*}{Remark}
		\newcommand{\myremarkautorefname}{remark}
%Det är för att få mellanrum innan definitioner och liknande
\makeatletter
\def\thm@space@setup{%
	\thm@preskip=\parskip \thm@postskip=0pt
}
\makeatother

%%EGNA KOMMANDON
\usepackage{xifthen} %Logik i kommdon
\usepackage{ifthen}

%Simplyfing - det är enbart kommandon som gör livet lite lättare
\newcommand{\tover}[1]{\overline{\overline{#1}}} 
\newcommand{\mb}[1]{\mathbf{#1}} 
\newcommand{\mbb}[1]{\boldsymbol{#1}} 			%This is the almost the same as \mb but the behaviour of them differ for example the letter (small) f. 
\newcommand{\mbo}[1]{\mathbf{\overline{#1}}} 
\newcommand{\lv}[2]{\mathbf{#1}_{\{#2\}}} 

%Om man vill ha stor bokstav i autoref (Paketet används inte längre utan är enbart för bakåt kompabilitet)
\newcommand{\Autoref}[1]{\capitalisewords{\autoref{#1}}} 

%%MATH - SETS 
\newcommand{\RR}[1][]{\ifthenelse{\equal{#1}{}}{\mathbb{R}}{\mathbb{R}^{#1}}}
\newcommand{\NN}[1][]{\ifthenelse{\equal{#1}{}}{\mathbb{N}}{\mathbb{N}^{#1}}}
\newcommand{\ZZ}[1][]{\ifthenelse{\equal{#1}{}}{\mathbb{Z}}{\mathbb{Z}^{#1}}}
\newcommand{\CC}[1][]{\ifthenelse{\equal{#1}{}}{\mathbb{C}}{\mathbb{C}^{#1}}}

%%MATH - OPERATORS
%\newcommand{\PP}[1]{\operatorname{P}\left(#1\right)}
\newcommand{\PP}[1][]{\ifthenelse{\equal{#1}{}}{\operatorname{P}}{\operatorname{P}\left(#1\right)}}
\newcommand{\PQ}[1][]{\ifthenelse{\equal{#1}{}}{\operatorname{Q}}{\operatorname{Q}\left(#1\right)}}

\newcommand{\diag}[1]{\operatorname{diag}\left(#1\right)}
\newcommand{\EE}[1]{\operatorname{E}\left[#1\right]}
\newcommand{\EEb}[2]{\operatorname{E}\left[ \left.#1\right|#2 \right]}
\newcommand{\EEE}[3]{\operatorname{E}_{#2}^{\operatorname{#3}}\left[#1\right]}
\newcommand{\EEEb}[3]{\operatorname{E}^{\operatorname{#3}}\left[\left.#1\right|#2\right]}

\newcommand{\V}[1]{\operatorname{Var}\left[#1\right]}
\newcommand{\tr}[1]{\operatorname{Tr}\left[#1\right]}
\newcommand{\TT}{{\intercal}}
\newcommand{\Ordo}[1]{\mathcal{O}(#1)}
\newcommand{\ordo}[1]{\textit{o}(#1)}

\newcommand{\I}[1]{\operatorname{\mathds{1}}_{\left\{#1 \right\}}}
\newcommand{\abs}[1]{\left|#1\right|}
\newcommand{\norm}[2]{\left|\left|#1\right|\right|_{#2}}

%%KORTVERSIONER AV NAMN TILL OLIKA SPECIELLA BOKSTÄVSSTILAR
%Operatorname
\newcommand{\Ao}{{\operatorname{A}}}
\newcommand{\Bo}{{\operatorname{B}}}
\newcommand{\Co}{{\operatorname{C}}}
\newcommand{\Do}{{\operatorname{D}}}
\newcommand{\Eo}{{\operatorname{E}}}
\newcommand{\Fo}{{\operatorname{F}}}
\newcommand{\Go}{{\operatorname{G}}}
\newcommand{\Ho}{{\operatorname{H}}}
\newcommand{\Io}{{\operatorname{I}}}
\newcommand{\Jo}{{\operatorname{J}}}
\newcommand{\Ko}{{\operatorname{K}}}
\newcommand{\Lo}{{\operatorname{L}}}
\newcommand{\Mo}{{\operatorname{M}}}
\newcommand{\No}{{\operatorname{N}}}
\newcommand{\Oo}{{\operatorname{O}}}
\newcommand{\Po}{{\operatorname{P}}}
\newcommand{\Qo}{{\operatorname{Q}}}
\newcommand{\Ro}{{\operatorname{R}}}
\newcommand{\So}{{\operatorname{S}}}
\newcommand{\To}{{\operatorname{T}}}
\newcommand{\Uo}{{\operatorname{U}}}
\newcommand{\Vo}{{\operatorname{V}}}
\newcommand{\Wo}{{\operatorname{W}}}
\newcommand{\Xo}{{\operatorname{X}}}
\newcommand{\Yo}{{\operatorname{Y}}}
\newcommand{\Zo}{{\operatorname{Z}}}

%%Mathcal
\newcommand{\Ac}{{\mathcal{A}}}
\newcommand{\Bc}{{\mathcal{B}}}
\newcommand{\Cc}{{\mathcal{C}}}
\newcommand{\Dc}{{\mathcal{D}}}
\newcommand{\Ec}{{\mathcal{E}}}
\newcommand{\Fc}{{\mathcal{F}}}
\newcommand{\Gc}{{\mathcal{G}}}
\newcommand{\Hc}{{\mathcal{H}}}
\newcommand{\Ic}{{\mathcal{I}}}
\newcommand{\Jc}{{\mathcal{J}}}
\newcommand{\Kc}{{\mathcal{K}}}
\newcommand{\Lc}{{\mathcal{L}}}
\newcommand{\Mc}{{\mathcal{M}}}
\newcommand{\Nc}{{\mathcal{N}}}
\newcommand{\Oc}{{\mathcal{O}}}
\newcommand{\Pc}{{\mathcal{P}}}
\newcommand{\Qc}{{\mathcal{Q}}}
\newcommand{\Rc}{{\mathcal{R}}}
\newcommand{\Sc}{{\mathcal{S}}}
\newcommand{\Tc}{{\mathcal{T}}}
\newcommand{\Uc}{{\mathcal{U}}}
\newcommand{\Vc}{{\mathcal{V}}}
\newcommand{\Wc}{{\mathcal{W}}}
\newcommand{\Xc}{{\mathcal{X}}}
\newcommand{\Yc}{{\mathcal{Y}}}
\newcommand{\Zc}{{\mathcal{Z}}}

%%MathS\mathscr
\newcommand{\Ascr}{{\mathscr{A}}}
\newcommand{\Bscr}{{\mathscr{B}}}
\newcommand{\Cscr}{{\mathscr{C}}}
\newcommand{\Dscr}{{\mathscr{D}}}
\newcommand{\Escr}{{\mathscr{E}}}
\newcommand{\Fscr}{{\mathscr{F}}}
\newcommand{\Gscr}{{\mathscr{G}}}
\newcommand{\Hscr}{{\mathscr{H}}}
\newcommand{\Iscr}{{\mathscr{I}}}
\newcommand{\Jscr}{{\mathscr{J}}}
\newcommand{\Kscr}{{\mathscr{K}}}
\newcommand{\Lscr}{{\mathscr{L}}}
\newcommand{\Mscr}{{\mathscr{M}}}
\newcommand{\Nscr}{{\mathscr{N}}}
\newcommand{\Oscr}{{\mathscr{O}}}
\newcommand{\Pscr}{{\mathscr{P}}}
\newcommand{\Qscr}{{\mathscr{Q}}}
\newcommand{\Rscr}{{\mathscr{R}}}
\newcommand{\Sscr}{{\mathscr{S}}}
\newcommand{\Tscr}{{\mathscr{T}}}
\newcommand{\Uscr}{{\mathscr{U}}}
\newcommand{\Vscr}{{\mathscr{V}}}
\newcommand{\Wscr}{{\mathscr{W}}}
\newcommand{\Xscr}{{\mathscr{X}}}
\newcommand{\Yscr}{{\mathscr{Z}}}
\newcommand{\Zscr}{{\mathscr{Y}}}

%%Differential opertorer
%\newcommand{\dx}{\operatorname{d}} %för att anvädna i integraler flr att få ett rakt d
\newcommand{\dx}[1][]{\mathrm{d}^{#1}\!} 
\newcommand{\dxx}[1][]{\dx \text{#1}}
\newcommand{\Rp}[1][]{\ifthenelse{\equal{#1}{}}{\operatorname{Re}}{\operatorname{Re}\left\{#1\right\} }}
\newcommand{\Ip}[1][]{\ifthenelse{\equal{#1}{}}{\operatorname{Im}}{\operatorname{Im}\left\{#1\right\} }}
\DeclareMathOperator{\Log}{\operatorname{Log}}
\DeclareMathOperator{\Ln}{\operatorname{Ln}}
\DeclareMathOperator{\Arg}{\operatorname{Arg}}

%Lite of sannolikhets och sigma-algebror
\newcommand{\generator}[2]{{\mathcal{#1}(\mathcal{#2})}}
\newcommand{\Bor}[1][]{\ifthenelse{\equal{#1}{}}{\mathcal{B}\left(\mathbb{R}\right)}{\mathcal{B}\left(\mathbb{R}^{#1}\right)}}

\newcommand{\cc}[0]{\mathsf{c}}
\newcommand{\prightarrow}{\stackrel{p}{\rightarrow}}
\newcommand{\arightarrow}{\stackrel{a.s.}{\rightarrow}}
\newcommand{\lprightarrow}[1]{\stackrel{L_{#1}}{\rightarrow}}
\newcommand{\ninfty}{n\rightarrow\infty}
\newcommand{\eqd}{\stackrel{d}{=}}

\DeclareMathOperator*{\argmax}{arg\,max}
\DeclareMathOperator*{\argmin}{arg\,min}
\DeclareMathOperator*{\arcsinh}{arcsinh}
\DeclareMathOperator*{\arccosh}{arccosh}
\DeclareMathOperator*{\arctanh}{arctanh}

%%COMMENTS
\newcommand{\furl}[1]{\footnote{\url{#1}}}
\newcommand{\tmcomment}[2]{	\ifthenelse{\equal{#1}{t}}{\left\{\text{#2}\right\}}{\left\{#2\right\}}	}

\newcommand*{\crom}[1]{\circled{\rom{#1}}}


%%ÖVRIGT
\newcommand*{\vertbar}{\rule[-1ex]{0.5pt}{2.5ex}}	%Detta är för att kunna rita fina streck i Matri
\newcommand*{\horzbar}{\rule[.5ex]{4.5ex}{0.5pt}}
\newcommand*{\ee}[1]{\cdot10^{#1}}
%\newcommand*{\e}[1]{{\mathrm{e}^{#1}}}

\newcommand{\e}[1][]{%
\ifthenelse{\equal{#1}{}}{\mathrm{e}}{\mathrm{e}^{#1}}%
}



\newcommand{\rom}[1]{\textup{\uppercase\expandafter{\romannumeral#1}}} %Göra Romerska siffror
\newcommand*\circled[1]{\tikz[baseline=(char.base)]{\node[shape=circle,draw,inner sep=2pt] (char) {#1};}} %Få cirklar kring bnokstäver och siffror

\newcommand\Tstrut{\rule{0pt}{2.6ex}}         % = `top' strut
\newcommand\Bstrut{\rule[-0.9ex]{0pt}{0pt}}   % = `bottom' strut
